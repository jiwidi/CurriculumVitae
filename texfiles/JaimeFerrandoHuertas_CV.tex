%%%%%%%%%%%%%%%%%
% This is an example CV created using altacv.cls (v1.1.5, 1 December 2018) written by
% LianTze Lim (liantze@gmail.com), based on the
% Cv created by BusinessInsider at http://www.businessinsider.my/a-sample-resume-for-marissa-mayer-2016-7/?r=US&IR=T
%


\documentclass[10pt,a4paper,ragged2e]{texfiles/altacv}

%% AltaCV uses the fontawesome and academicon fonts
%% and packages.
%% See texdoc.net/pkg/fontawecome and http://texdoc.net/pkg/academicons for full list of symbols. You MUST compile with XeLaTeX or LuaLaTeX if you want to use academicons.

% Change the page layout if you need to
\geometry{left=1cm,right=11cm,marginparwidth=8.8cm,marginparsep=1.2cm,top=1.25cm,bottom=1.25cm}

% Change the font if you want to, depending on whether
% you're using pdflatex or xelatex/lualatex
\ifxetexorluatex
  % If using xelatex or lualatex:
  \setmainfont{Carlito}
\else
  % If using pdflatex:
  \usepackage[utf8]{inputenc}
  \usepackage[T1]{fontenc}
  \usepackage[default]{lato}
\fi


\usepackage{hyperref}

% Change the colours if you want to
\definecolor{NeonGreen}{HTML}{1DB954}
\definecolor{SlateGrey}{HTML}{2E2E2E}
\definecolor{LightGrey}{HTML}{37474F}
\colorlet{heading}{NeonGreen}
\colorlet{accent}{NeonGreen}
\colorlet{emphasis}{SlateGrey}
\colorlet{body}{LightGrey}

% Change the bullets for itemize and rating marker
% for \cvskill if you want to
\renewcommand{\itemmarker}{{\small\textbullet}}
\renewcommand{\ratingmarker}{\faCircle}

%% sample.bib contains your publications
\addbibresource{sample.bib}

\begin{document}
\name{Jaime Ferrando Huertas}
\tagline{Machine Learning engineer}
% Cropped to square from https://en.wikipedia.org/wiki/Marissa_Mayer#/media/File:Marissa_Mayer_May_2014_(cropped).jpg, CC-BY 2.0
\personalinfo{%
  % Not all of these are required!
  % You can add your own with \printinfo{symbol}{detail}
    \href{mailto:fhjaime96@gmail.comm}{\email{fhjaime96@gmail.com}}
    % \phone{+460706164458}
    \location{Stockholm, Sweden}
    % \linkedin{https://www.linkedin.com/in/jaime-ferrando-huertas-611ab5130/}
    \href{https://github.com/jiwidi}{\github{github.com/jiwidi}}
    \href{https://imjai.me}{\location{imjai.me}}
%   \orcid{orcid.org/0000-0000-0000-0000} % Obviously making this up too. If you want to use this field (and also other academicons symbols), add "academicons" option to \documentclass{altacv}
}

%% Make the header extend all the way to the right, if you want.
\begin{fullwidth}
\makecvheader
\end{fullwidth}

%% Depending on your tastes, you may want to make fonts of itemize environments slightly smaller
\AtBeginEnvironment{itemize}{\small}

%% Provide the file name containing the sidebar contents as an optional parameter to \cvsection.
%% You can always just use \marginpar{...} if you do
%% not need to align the top of the contents to any
%% \cvsection title in the "main" bar.

\cvsection[texfiles/page1sidebar]{Work Experience}

\cvevent{Founder engineer}{Shaped}{April 2022 -- Now}{NY, USA}{
\cvtag{AWS} 
\cvtag{Python}
\cvtag{YCombinator}
\cvtag{Pytorch}
}
\begin{itemize}
    \item Developing and implementing machine learning models and infrastructure to enable customers to build highly effective, real-time recommendation systems.
\end{itemize}
\vspace{4px}
\cvevent{Machine learning engineer}{H\&M}{Feb 2020 -- Now}{Stockholm, Sweden}{
\cvtag{PySpark} 
\cvtag{Python}
\cvtag{Azure}
\cvtag{Pytorch}
\cvtag{GCP}}
\begin{itemize}
    \item Host first H\&M \href{https://www.kaggle.com/competitions/h-and-m-personalized-fashion-recommendations}{\textbf{Kaggle competition}} with over, 2900 participating teams and 50.000 dollars in prices.
    \item \textbf{Lead} neural based recommendation systems at H\&M. Wrote first neural based system, first session based recommended, first two-tower ranking model and first online serving model. Increasing customer retention, revenue, and interactions across online services.
    \item Develop engineering infrastructure for live model serving at GCP for in session use of recommendation models.
    \item Design live a/b testing for different models as well as feedback loop for internal dashboards.
\end{itemize}

\vspace{4px}

\cvevent{Data scientist}{Sandvik}{Oct 2018 -- Feb 2020}{Stockholm, Sweden}{
\cvtag{Google Cloud} 
\cvtag{Python}}
\begin{itemize}
    \item Develop, deploy and maintain abnormal behavior detection models for multivariate industrial systems.
\end{itemize}
\vspace{10px}

\cvevent{Data Scientist}{Polystar Group  }{Oct 2017 -- Jul 2018}{Stockholm, Sweden}{
\cvtag{C++} 
\cvtag{Python}
\cvtag{Scala}}
\begin{itemize}
    \item Bachelor thesis for abnormal behavior detection in telecommunications networks.
\end{itemize}
\vspace{4px}

\cvevent{Software engineer}{Ahora Freeware  }{Jun 2016 -- Jul 2017}{Valencia, Spain}{\cvtag{C\#}\cvtag{.NET}}
\begin{itemize}
\end{itemize}


\vspace{1px}

\cvsection{Education}

\cvevent{Master's degree in Artificial Intelligence, Pattern Recognition and Digital Imaging}{UPV, Politecnic University of Valencia}{Sept 2020 -- Jun 2021}{Valencia, Spain}{}
\cvevent{Exchange program master courses}{KTH, Kungliga Tekniska Högskolan}{Sept 2017 -- June 2018}{Stockholm, Sweden}{}
\cvevent{Bachelor in Computer Science}{UPV, Politecnic University of Valencia}{Sept 2014 -- June 2018}{Valencia, Spain}{}{Machine learning specialization. \\
High performance group.}


% \cvsection{Languages}
% English: Bilingual \textbf{|} Spanish: Native \textbf{|} Catalan: Native


\newpage


% \cvsection{Public Projects}

% \textcolor{SlateGrey}{\textbf{\href{https://github.com/jiwidi/las-pytorch}{\faGithub \, LAS-Pytorch }}} \\ 
% Implementation of Listen, Attend and spell model for E2E ASR.
% \newline
% \vspace{1pt}

% \textcolor{SlateGrey}{\textbf{\href{https://github.com/jiwidi/MASTER_THESIS/blob/master/thesis.pdf}{\faGithub \,  Neural recommender systems at H\&M  }}} \\ 
% Master Thesis (Cum-laude graded) for Neural based recommender at H\&M. Proposed models replaced production models and are currently in production.
% \newline
% \vspace{1pt}

% \textcolor{SlateGrey}{\textbf{\href{https://github.com/jiwidi/time-series-forecasting-with-python}{\faGithub \,  Time series with Python  }}} \\ 
% A use-case focused tutorial for time series forecasting with Python.
% \newline
% \vspace{1pt}


% \textcolor{SlateGrey}{\textbf{\href{https://github.com/jiwidi/jupyter-lab-docker-rpi}{\faGithub \, Jupyter-lab-docker-rpi}}} \\ 
% A Docker image to run JupyterLab on your Raspberry Pi. 
% \newline
% \vspace{1pt}

\phantom{Sklearn, pandas, numpy, python, scala, machine learning, data science, ai, artificial intelligence, please dont read this, cloud computing, elastic, aws, amazon web services, azure, gcp, google cloud, scala, java, c, c++, business intelligence, BI, data engineering, buzzwords so recruits wont filter this cv out, warehouse, apache, Supervised learning, unsupervised learning, online learning, deep learning, algorithms, regression, classification, accuracy, recall, precision, f1, container, linux, tcp, big data, analytics, bachelor, master, paper, engineering, facebook, google, apple, amazon, netflix, instagram, whatsapp, faang }



\end{document}